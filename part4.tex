\section{Part 4 - Observability}
\subsection{Task A}

Matrix A, B, C, and D can be found by inserting equations 1a-1f into $\bm{\dot{x}} = \bm{Ax} + \bm{B}u + \bm{Ew}$ and $y = \bm{Cx} + v$.

\begin{equation}
    \bm{A} =
    \begin{bmatrix}
        0 & 1 & 0 & 0 & 0 \\
        -\omega_0^2 & -2\lambda\omega_0 & 0 & 0 & 0 \\
        0 & 0 & 0 & 1 & 0 \\
        0 & 0 & 0 & -\frac{1}{T} & -\frac{K}{T} \\
        0 & 0 & 0 & 0 & 0
    \end{bmatrix} \quad \bm{B} =
    \begin{bmatrix}
        0 \\
        0 \\
        0 \\
        \frac{K}{T} \\
        0
    \end{bmatrix}
\end{equation}

\begin{equation}
    \bm{C} = 
    \begin{bmatrix}
        0 & 1 & 1 & 0 & 0
    \end{bmatrix} \qquad \bm{E} =
    \begin{bmatrix}
        0 & 0 \\
        K\omega & 0 \\
        0 & 0 \\
        0 & 0 \\
        0 & 1
    \end{bmatrix}
\end{equation}

\subsection{Task B}
To check if the system is observable without disturbances, we set wave and current disturbance to zero. \\ Wave disturbance: $\psi_\omega = 0$ which gives $\dot{\psi_\omega} = 0$ and $\dot{\xi} = \psi_\omega = 0$. \\ Current disturbance: $\omega_b = 0$  which gives $\dot{b} = 0$

\begin{equation}
    \text{Our new state vector becomes} \quad
    \bm{x} = 
    \begin{bmatrix}
        \psi \\     \nonumber
        r
    \end{bmatrix} \quad \text{since the other states are 0.}
\end{equation}

The new $\bm{A}$ and $\bm{C}$ becomes:

\begin{equation}
    \bm{A} = 
    \begin{bmatrix}
        0 & 1 \\
        0 & -\frac{1}{T}
    \end{bmatrix} \quad y = \psi + v \quad \Rightarrow \quad \bm{C} =
    \begin{bmatrix}
        1 & 0
    \end{bmatrix}
\end{equation}

The observability matrix is calculated using $\bm{A}$ and $\bm{C}$: 

\begin{equations}
    \begin{center}
        \bm{$O$} =
        \begin{bmatrix}
            C \\
            CA
        \end{bmatrix} = 
        \begin{bmatrix}
            1 & 0 \\
            0 & 1
        \end{bmatrix}
    \end{center}
\end{equations}

This observability matrix has full rank, so the system without disturbances is observable. \\

\subsection{Task C}
By setting $\psi_\omega = 0$ and $\omega_b \neq 0$, from the new state vector $\bm{x}$, $\bm{A}$ and $\bm{C}$ becomes:

\begin{equation}
    \bm{x} =
    \begin{bmatrix}
        \psi \\
        r \\
        b
    \end{bmatrix} \quad \text{gives} \quad
    \bm{A} =
    \begin{bmatrix}
        0 & 1 & 0 \\
        0 & -\frac{1}{T} & -\frac{K}{T} \\
        0 & 0 & 0
    \end{bmatrix} \quad \text{and} \quad
    \bm{C} =
    \begin{bmatrix}
        1 & 0 & 0
    \end{bmatrix}
\end{equation} \\


Observability matrix for a 3x3 matrix is obtained in this way:

\begin{equations}
    \begin{center}
        \bm{$O$} =
        \begin{bmatrix}
            C \\
            CA \\
            CA^2
        \end{bmatrix} = 
        \begin{bmatrix}
            1 & 0 & 0 \\
            0 & 1 & 0 \\
            0 & -\frac{1}{T} & -\frac{K}{T}
        \end{bmatrix} \text{row reduction} \Rightarrow
        \begin{bmatrix}
            1 & 0 & 0 \\
            0 & 1 & 0 \\
            0 & 0 & 1
        \end{bmatrix}
    \end{center}
\end{equations}\\
\\

As seen, our observability matrix with current disturbance and not wave disturbance does have full rank, which implies that this system is observable.

\subsection{Task D}
By setting $\psi_\omega \neq 0$, $\omega_b = 0$ we can check if the system is observable with wave disturbance and without current disturbance. The new state vector gives the following $\bm{A}$ and $\bm{C}$ matrices.

\begin{equation}
    \bm{x} =
    \begin{bmatrix}
        \xi_\omega \\
        \psi_\omega \\
        \psi \\
        r
    \end{bmatrix} \quad \text{gives} \quad
    \bm{A} =
    \begin{bmatrix}
        0 & 1 & 0 & 0 \\
        -\omega_0^2 & -2\lambda\omega_0 & 0 & 0 \\
        0 & 0 & 0 & 1 \\
        0 & 0 & 0 & -\frac{1}{T}
    \end{bmatrix} \quad \text{and} \quad
    \bm{C} =
    \begin{bmatrix}
        0 & 1 & 1 & 0   \nonumber
    \end{bmatrix}
\end{equation} \\

The observability matrix for task d becomes:

\begin{equations}
    \begin{center}
        \bm{$O$} =
        \begin{bmatrix}
            C \\
            CA \\
            CA^2 \\
            CA^3
        \end{bmatrix} = 
        \begin{bmatrix}
            0 & 1 & 1 & 0 \\
            -\omega_0^2 & -2\lambda\omega_0 & 0 & 1 \\
            2\lambda\omega_0^3 & (4\lambda^2-1)\omega_0^2 & 0 & -\frac{1}{T} \\
            (1-4\lambda^2)\omega_0^4 & (2 - 4\lambda^2)2\lambda\omega_0^3 & 0 & \frac{1}{T^2}
        \end{bmatrix}
    \end{center}
\end{equations}\\
\\

The rank of this observability matrix was found by using the matlab function rank(\bm{$O$}), which gave rank(\bm{$O$}) = 4. By this result we can conclude that the system is observable with wave disturbance.

\subsection{Task E}
To check if the system is observable with wave and current disturbances, A and C matrices is unchanged:

\begin{equation}
    \bm{A} = 
    \begin{bmatrix}
        0 & 1 & 0 & 0 & 0 \\
        -\omega_0^2 & -2\lambda\omega_0 & 0 & 0 & 0 \\
        0 & 0 & 0 & 1 & 0 \\
        0 & 0 & 0 & -\frac{1}{T} & -\frac{K}{T} \\
        0 & 0 & 0 & 0 & 0
    \end{bmatrix}
\end{equation}

\begin{equation}
    \bm{$O$} =
    \begin{bmatrix}
        C \\
        CA \\
        CA^2 \\
        CA^3 \\
        CA^4
    \end{bmatrix} =
    \begin{bmatrix}
        0 & 1 & 1 & 0 & 0 \\
        -\omega_0^2 & -2\lambda\omega_0 & 0 & 1 & 0 \\
        2\lambda\omega_0^3 & 4\lambda^2\omega_0^2 & 0 & -\frac{1}{T} & -\frac{K}{T} \\
        -4\lambda^2\omega_0^4 & -8\lambda^3\omega_0^3 & 0 & \frac{1}{T^2} & \frac{K}{T^2} \\
        8\lambda^3\omega_0^5 & 16\lambda^4\omega_0^4 & 0 & -\frac{1}{T^3} & -\frac{K}{T^3} \quad
    \end{bmatrix}
\end{equation}























